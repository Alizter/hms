\documentclass{article}
\usepackage{graphicx}

\usepackage{amsmath}
\usepackage{amsfonts}
\usepackage{amsthm}
\usepackage{tikz-cd}

\usepackage{verbatim}
\usepackage[utf8]{inputenc}
\usepackage[english]{babel}

\usepackage[colorinlistoftodos]{todonotes}

\newcommand{\loops}{\Omega}
\newcommand{\susp}{\Sigma}
\newcommand{\N}{\mathbb{N}}

\theoremstyle{definition}
\newtheorem{theorem}{Theorem}[section]
\newtheorem{defin}[theorem]{Definition}
\newtheorem{lemma}[theorem]{Lemma}
\newtheorem{cor}[theorem]{Corollary}
\newtheorem{prop}[theorem]{Proposition}
\newtheorem{remark}[theorem]{Remark}

\newcommand{\pr}{\ensuremath{\mathsf{pr}}}
\newcommand{\hfiber}{\ensuremath{\mathsf{hfiber}}}
\newcommand{\concat}{\ensuremath{\mathsf{concat}}}


\title{Hilton-Milnor Splitting in HoTT}
\author{Ali Caglayan}

\begin{document}
    
    \maketitle
    \todo{Title okay?}
    

    \begin{abstract}
        \todo{Abstract needs to be fleshed out}
        We show in homotopy type theory that for pointed types $X, Y$ we have a pointed equivalence:
        $\loops (X \vee Y) \simeq \loops X \times \loops Y \times \loops \susp (\loops X \wedge \loops Y)$
        known as the Hilton-Milnor splitting.
        \begin{comment}
         Observing that this can be iterated we can observe that:
            \begin{equation*}
                \loops \susp (X \vee Y) \simeq \loops \susp X \times \loops \susp Y \times \loops \susp
                    \left(
                        \bigvee_{i, j : \N} X^{\wedge i} \wedge Y^{\wedge j}
                    \right)
            \end{equation*}
        \end{comment}
    \end{abstract}

    \section{Introduction}

    \todo{something about classical homotopy theory}

\cite{hottbook}


    \section{Homotopy pullbacks and descent}
    
    

\begin{lemma}[Splitting lemma]
    Let $f : X \to_* Y$ be a pointed map. Let $g : \loops Y \to_* \loops X$ be a section of $\loops f$. Then the composition of pointed maps
    
    \begin{equation*}
        \loops B \times \loops \hfiber(f) \xrightarrow{g \times \mathsf{pr}_1}_*
            \loops A \times \loops A \xrightarrow{\concat}_* \loops A    
    \end{equation*}
    is an equivalence of pointed types.

\end{lemma}


% Cite Rijke Theorem 2.2.11
\begin{theorem}[Descent]
    \todo{Statement and version for pointed types}
    \begin{equation*}
        \begin{tikzcd}[row sep=scriptsize, column sep=scriptsize]
            A \ar[dr] \ar[rr] \ar[dd] & & B \ar[dr] \ar[dd] \\
            & C \ar[rr, crossing over] & & D \ar[dd] \\
            A' \ar[dr] \ar[rr] & & B' \ar[dr] \\
            & C' \ar[rr] \ar[from=uu, crossing over] & & D' \\
        \end{tikzcd}
    \end{equation*}

\end{theorem}


\begin{lemma}[Two-pullbacks lemma]
    Let \todo{Needs proof and proper statement. Also a note about pointed version. }
    \begin{equation*}
        \begin{tikzcd}[row sep=large, column sep=large]
            A \ar[r] \ar[d] & B \ar[r] \ar[d] & C \ar[d] \\
            D \ar[r] & E \ar[r] & F \\
        \end{tikzcd}
    \end{equation*}
\end{lemma}

\begin{defin}[Wedge sum]
    Let $X, Y$ be pointed types. Define the \emph{wedge sum} (or simply \emph{wedge}) of $X$ and $Y$, denoted $X \vee Y$ to be the following pushout:
    \begin{equation*}
        \begin{tikzcd}[row sep=large, column sep=large]
            1 \ar[r] \ar[d] \ar[dr, phantom, "\ulcorner", very near start] & X \ar[d] \\
            Y \ar[r] & X \vee Y
        \end{tikzcd}
    \end{equation*}
    \todo{Wedge induction, recursion, "constructors"}
\end{defin}

\begin{defin}[Wedge inclusion]
    There is a map $\omega : X \vee Y \to X \times Y$ called the wedge inclusion. Defined by wedge recursion:
    \begin{itemize}
        \item $\mathsf{winl}(x) \mapsto (x, *)$
        \item $\mathsf{winr}(y) \mapsto (*, y)$
        \item $\mathsf{wglue} \mapsto \mathsf{refl}_{(*,*)}$
    \end{itemize}
\end{defin}

% Cite Devalapurkar-Haine
\begin{lemma}[Fiber of wedge inclusion]
    There is a pointed equivalence
    \begin{equation*}
        \hfiber(\omega) \simeq \loops X * \loops Y
    \end{equation*}
\end{lemma}

\begin{proof}
    We have zero maps $\loops X \to Y$ and $\loops Y \to X$. We have maps $\loops Y \to \hfiber(\omega)$ which takes $p : \loops Y$ and maps it to $(*, \mathsf{pair}^= (\mathsf{refl}, p))$ and likewise for $\loops X \to \hfiber(\omega)$.
    Observe that the following diagram commutes:
    \begin{equation*}
        \begin{tikzcd}
            \loops X \times \loops Y \ar[rr, "\pr_1"] \ar[dd] \ar[rd, "\pr_2"] & & \loops X \ar[dd] \ar[rd] \\
            & \loops Y \ar[rr, crossing over] & & \hfiber(\omega) \ar[rr] \ar[dd] & & 1 \ar[dd] \\
            1 \ar[rr] \ar[rd] & & Y \ar[rd, "\mathsf{winr}"] \\
            & X \ar[from=uu, crossing over] \ar[rr, "\mathsf{winl}"] & & X \vee Y \ar[rr, "\omega"] & & X \times Y \\
        \end{tikzcd}
    \end{equation*}
    Now the left and back faces of the cube are pullbacks. The bottom face is a pushout by definition. The front and right faces are pullbacks since the commutative square joined with the right square is a pullback by definition, therefore by the pullback lemma they are pullbacks. Hence by descent the top square is a pushout. Observe that the pushout of the top square is the join $\loops X * \loops Y$ hence $\hfiber(\omega) \simeq \loops X * \loops Y$.
\end{proof}

\begin{lemma}[Join as a suspended smash]
    We have the following pointed equivalence for pointed types $X$ and $Y$:
    \begin{equation*}
        X * Y \simeq \susp (X \wedge Y)
    \end{equation*}
\end{lemma}

\begin{remark}
    Thus $\hfiber(\omega) \simeq \susp (\loops X \wedge \loops Y)$
\end{remark}

    
    \section{The Hilton-Milnor splitting}

    


\begin{theorem}[Hilton-Milnor splitting]
    Let $X, Y$ be pointed types. Then there is a pointed equivalence:
    \begin{equation*}
        \loops (X \vee Y) \simeq \loops X \times \loops Y \times \loops \susp (\loops X \wedge \loops Y)
    \end{equation*}
\end{theorem}


\end{document}
